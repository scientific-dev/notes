\documentclass[12pt]{article}
\usepackage{graphicx}
\usepackage[colorlinks=true, pdfborder={0 0 0}, linkcolor=blue, urlcolor=blue, citecolor=blue]{hyperref}
\usepackage[a4paper, margin=.8in]{geometry}
\usepackage{float}
\usepackage{amsmath}
\usepackage{amssymb}
\usepackage{anysize}
\usepackage{caption}
\usepackage{tcolorbox}

\begin{document}

\fontsize{14}{16}
\selectfont
\captionsetup[figure]{font=normal}
\newtcolorbox{note}[1][]{colback=white, colframe=black, sharp corners, boxrule=0.3mm, #1}

\title{Lasers}
\author{Sudarsan D Naidu}
\maketitle

\tableofcontents

\section{Introduction}

LASER means \textbf{Light Amplification by Stimulated Emission of Radiation}. To understand the mechanism of Laser, one has to understand the concept of Energy levels (which will be studied later in Quantum Field theory).

\subsection{Energy Level}

An energy level is a specific, quantized state that an $e^{-}$ can occupy in an atom or molecule, each with a fixed energy value. We will be studying about these energy levels in depth when we will be learning Quantum Physics. \vspace{.2cm}

\textbf{What does an atom getting excited mean?} An atom getting excited means that the atom has gained some energy (relative to its ground state). When an electron of the atom get excited, it gains energy and as an electron is part of an atom, so the atom too does gain energy. From whatever we know till now, when a photon is absorbed by an atom, it is more accurate to say that the electron has been excited. But, in this chapter, we will often keep atom as our point of reference. So, whenever we say that an atom has been excited, we refer to excitation of the outermost valence shell electron. Refer to Born-Oppenheimer approximation for a better idea on how nucleus of an atom is not affected when the photon is absorbed by the atom.

\begin{quote}
    If an electron has been excited, then the atom too gets excited but if an atom has been excited, there are many possibilities. But, in this chapter, we will be only considering the excitation of the outermost valence shell electron as the only possibility.
\end{quote}

\subsection{Energy Gap}

Usually for an \textbf{isolated gaseous atom}, there are many energy levels in discrete which the electron can occupy. But, in the case of solids, due to tightly packed atoms and their orbital overlap, the discrete energy levels become very close and appear to be continuous. Such continuous energy levels are called as energy bands and the $e^{-}$ can take any energy value between the lowest and highest value of energy band. The concept of energy bands will be discussed again in Band Theory of Solids in depth.

\subsection{Classical Representation of Atom}

In classical view (an extended version of Newtonian mechanics), these energy levels are related with the orbits of the $e^{-}$ revolving around the nucleus (See figure 1).

\begin{figure}[H]
    \centering
    \includegraphics[scale=0.8]{./img/01_classical_atom.png}
    \caption{Classical Representation of an Atom}
\end{figure}

This idea is very outdated and is wrong as we move towards the concept of Quantum Mechanics. The atom is often represented like this in some books to keep things simple. In the quantum mechanical model of the atom, we consider electrons to be in the orbitals corresponding to their energy levels.

\subsection{Energy Level Diagrams}

To explain phenomena related to interaction between radiation and matter, we need a good diagrammatic representation of an atom. As stated earlier, we are not interested in outdated classical representation of atom anymore in favour of the quantum mechanical model. \vspace{.2cm}

Now, if we follow the quantum mechanical model, representing orbitals in an atom is highly complicated, making it even more challenging to use those diagrams to explain certain phenomena. We have discussed earlier that in this chapter we are going to equate energy states of an atom with the energy state of its outermost valence electron, i.e. when an electron excites from $E_{1}$ to $E_{2}$, the atom too excites from $E_{1}$ to $E_{2}$. So, instead of using diagrams of electrons, orbitals or orbits, we will be using the diagram of energy levels (see figure 2) to explain several phenomena.

\begin{figure}[H]
    \centering
    \includegraphics[scale=0.8]{./img/02_energy_levels.png}
    \caption{Energy level diagram of an Atom}
\end{figure}

This energy level diagram helps use to visualise in which energy state the atom (or its outermost valence shell $e^{-}$) is in without violating any idea of quantum mechanical model of an atom. Compare figure 2 with figure 1.

\subsection{References}

\begin{itemize}
    \item Read Basavaraju's book on Engineering Physics for fundamental ideas (Chapter 5 Lasers)
    \item Watch this \href{https://www.youtube.com/watch?v=_JOchLyNO_w&t=233s}{YT video} for a complete animated video of the whole mechanism of LASER. The only con of this video will be its attempt of using the classical representation of atom in it.
    \item See this \href{https://www.reddit.com/r/Physics/comments/1ev7gss/are_energy_levels_for_electrons_or_atoms/}{Reddit post} for question on energy levels.
\end{itemize}

This page has to be updated once we complete studying Quantum Physics and its mechanics.

\section{Types of Interactions}

Working of laser depends on the three different ways of interaction of radiation and matter. In this chapter, there will only be an introduction to the interactions and will study them later in depth in Quantum Mechanics. \textbf{Note} that these interactions happen only in isolated atoms with discrete $E$ levels.

\subsection{Stimulated Absorption}

This interaction is also called as \textbf{Induced Absorption} in some books (like in Basavaraju's) while in some others, it is called as \textbf{Stimulated Absorption}.

\begin{figure}[H]
    \centering
    \includegraphics[scale=0.8]{./img/03_induced_absorption.png}
    \caption{Stimulated Absorption}
\end{figure}

Here (and for the other 2 interactions too), assume $E_{1}$ is the ground state and $E_{2}$ is some excited state of that atom.

\begin{equation}
    \Delta E = E_{2} - E_{1}
\end{equation}

\textbf{Note} although $E_{1}$ will be greater than $E_{2}$ in magnitude but their values are in negative thus making the $\Delta E$ positive. \vspace{.2cm}

Here, the photon is absorbed by the atom to excite itself from the ground state to an excited state. This is only possible if the energy of photon is same as the energy different between two levels. So, by \textbf{Planck's formula}.
\begin{equation}
    \Delta E = h\nu
\end{equation}

\subsection{Spontaneous Emission}

\begin{figure}[H]
    \centering
    \includegraphics[scale=0.8]{./img/04_spontaneous_emission.png}
    \caption{Spontaneous Emission}
\end{figure}

Any atom in an excited state is unstable and tends to de-excite to the ground state without any external aid (thus spontaneous). During this process, it emits an photon of an energy equal to the energy difference. \textbf{Note} that, there is an universal tendency for any system to be in the lowest energy state to keep itself stable.

Photons emitted by two atoms excited under (and are in) identical conditions undergoing spontaneous emission, \textbf{will not have any phase relationship} (i.e. in any direction and can have phase differences). So, photons emitted under this process are incoherent (i.e. no phase relationship). Such incoherent emissions can be seen in Thermoionic emission in bulb, candle flame, etc.

\subsection{Stimulated Emission}

\begin{figure}[H]
    \centering
    \includegraphics[scale=0.8]{./img/05_stimulated_emission.png}
    \caption{Stimulated Emission}
\end{figure}

Although, atoms can get de-excited without any external aid as seen in Spontaneous emission, they can also get de-excited by an incident photon of an energy equal to the energy difference between the excited levels. When such de-excitation happens, the atom emits another photon which will be in phase (i.e. coherent) with the incident photon. This process is also referred to as "negative absorption".

\section{Population distribution of Atoms in Excited states}

In a system, let $n_{i}$ be the number density (i.e. number of units per unit volume) of atoms existing in the energy state $i$. To keep things more simple in understanding the lasing process, we will use the energy level diagrams to show the distribution of atoms of system in the excited states (see figure 6). Also, by "population", we mean the "number" of atoms.

\begin{figure}[H]
    \centering
    \includegraphics[scale=.8]{./img/06_population_dist.png}
    \caption{Population distribution of atoms in the system}
\end{figure}

\textbf{Note} that population distribution is only for identical atoms. In real systems, the number of atoms exceed the number of energy levels possible in that system. (??)

\subsection{Rate of change of Population Density}

Then, $dn_{i}/dt$ will be the rate at which the number density of atoms exciting or de-exciting (depending on the sign of the derivative) from/to an energy state $i$. \vspace{.2cm}

Let's consider a system having energy states $E_{1}$, $E_{2}$, ..., $E_{i}$. Let it have a total population density (of atoms) $n$,
\begin{equation}
    n_{1} + n_{2} + ... + n_{i} = n
\end{equation}

We can differentiate this equation (w.r.t $t$) to get a relation between the rate of atoms exciting and de-exciting in each energy state,

\begin{equation*}
    \frac{dn_{1}}{dt} + \frac{dn_{2}}{dt} + ... + \frac{dn_{i}}{dt} = \frac{dn}{dt}
\end{equation*}

In a fixed system, we can assume $n$ to be constant, which makes $dn/dt$ to be 0,
\begin{equation*}
    \frac{dn_{1}}{dt} + \frac{dn_{2}}{dt} + ... + \frac{dn_{i}}{dt} = 0
\end{equation*} 

In transition of atoms, involving only two states, the rates of other states will become zero,
\begin{align}
    \frac{dn_{1}}{dt} + \frac{dn_{2}}{dt} & = 0 \notag \\
    \Rightarrow \frac{dn_{2}}{dt} & = -\frac{dn_{1}}{dt}
\end{align}

\subsection{Thermodynamic Equilibrium}

At Thermodynamic Equilibrium, the net energy exchanged will be 0 (energy exchange will happen but the sum will be 0). Excitation and de-excitation of atoms are energy exchanges too as they involve absorbing and emitting radiation (energy), so at Thermodynamic Equilibrium, the net change in the number of excited atoms will be 0 too. \vspace{.2cm}

This means, for any state $i$ at Thermodynamic Equilibrium, 

\begin{equation}
    \frac{dn_{i}}{dt} = 0
\end{equation}

\subsection{Boltzmann Factor}

At Thermodynamic Equilibrium, energy states having large population densities can be related with each other by the Boltzmann factor,

\begin{equation}
    \frac{n_{2}}{n_{1}} = e^{-\frac{E_{2} - E_{1}}{kT}}
\end{equation}

This formula is built on approximation of the theoretical data and works only for large $n$ values. \vspace{.2cm}

The derivation of this equation is given \href{https://www.phys.ufl.edu/~meisel/Boltzmann.pdf}{in this paper}.

\subsection{Non Thermodynamic Equilibrium Conditions}

We know in energy level scheme, $E_{2} > E_{1}$. By substituting this inequality in Boltzmann's equation in (6), we get $n_{2} < n_{1}$. If we place the system at a very huge temperature (i.e. $T \rightarrow \infty$ but in equilibrium), then from equation (6), we get $n_{2} \rightarrow n_{1}$. \vspace{.2cm}

As Boltzmann's equation is only valid at Thermodynamic Equilibrium, we can say $n_{2}$ is always lesser than or equal to $n_{1}$ (i.e $n_{2} \le n_{1}$) under Thermodynamic Equilibrium. In short, $n_{2}$ can never be greater than $n_{1}$ under Thermodynamic Equilibrium conditions. 
\begin{equation*}
    n_{2} \le n_{1} \text{ (under Thermodynamic Equilibrium)}
\end{equation*}

So, to make $n_{2} > n_{1}$, we have to feed energy in other ways to the system resulting in non equilibrium conditions which is essential for Lasing process (will be discussed later). 

\section{Einstein's Coefficients}

The formation of atomic spectral lines is due to the three different interactions of radiation and matter which was proposed by Einstein in 1916. In his studies, he assumed matter is in Thermodynamic Equilibrium (see 3.2) with Black Body radiation. \vspace{.2cm}

Each of these processes (interactions) proposed by Einstein are associated with an Einstein coefficient. Einstein considered the case of isotropic radiation of frequency $\nu$ (i.e. having same frequency in all directions the radiation is propagating at) and spectral energy density $\rho(\nu)$. \vspace{.2cm}

Read more about this \href{https://en.wikipedia.org/wiki/Einstein_coefficients}{in Wikipedia}.

\subsection{Spectral Energy Density}

Let radiation of continous spectrum of frequencies be incident on the system. Let $\rho(\nu)d\nu$ be the energy incident per unit volume (i.e. energy density) of the radiations having frequencies between $\nu$ and $\nu + d\nu$ in the spectrum. Here, $\rho(\nu)$ is called as spectral energy density (having dimensions of $E/A\nu$) which is the energy density per frequency of radiations having frequencies between $\nu$ and $\nu + d\nu$. \vspace{.2cm}

Also, in some books (like in Basavaraju's), this has been represented by $U_{\gamma}$ where the frequency is represented by $\gamma$. \textbf{Note} that, this parameter is independent of direction of radiation as we have assumed isotropic frequencies. \vspace{.2cm}

According to \textbf{Planck's law}, the spectral energy density is given as,

\begin{equation}
    \rho(\nu) = \frac{8\pi h\nu^3}{c^3} \frac{1}{e^\frac{h\nu}{kT} - 1}
\end{equation}

We will study about this law later in Quantum Physics.

\subsection{Description of Einstein's coefficients}

Let's consider two energy levels $E_{1}$ (ground state) and $E_{2}$ (some excited state) where $n_{1}$ and $n_{2}$ are the respective population densities of atoms in those energy states. Now, let's see how the three kinds of interactions of radiation and matter are described by the Einstein's coefficients,

\subsubsection{Stimulated Absorption}

In this process, the atoms excite from $E_{1}$ to $E_{2}$ due to the incident radiation of frequency $\nu$. To measure the rate of this process (and for the others too), we will consider $E_{1}$ (ground state) and its population density $n_{1}$ as reference, i.e.,
\begin{align*}
    \text{Rate of stimulated} & = \text{Rate at which atoms are undergoing stimulated} \\
    \text{absorption} & \quad \text{absorption from } E_{1} \text{ to } E_{2} \\ 
    & = (\frac{dn_{1}}{dt})_{st.a.}
\end{align*}

And, from experiments, we know, 
\begin{align*}
    \textnormal{Rate of stimulated absorption} \propto -n_{1}\rho(\nu) \\ 
    (\frac{dn_{1}}{dt})_{st.a.} \propto -n_{1}\rho(\nu)
\end{align*}

As atoms excite from $E_{1}$ to $E_{2}$, it results in decrease in $n_{1}$ (and increase in $n_{2}$) which explains the negative sign. The rate of stimulated absorption will be more if there are more atoms in the ground state $E_{1}$ to get excited to $E_{2}$ so the rate depends on $n_{1}$. This points out the fact that the rate of stimulated absorption will decrease as more and more atoms in $E_{1}$ get excited. Also, as this process is initiated by the energy of the incident radiation, it also depends on the spectral energy density of it. \vspace{.2cm}

The proportional constant for the equation is the Einstein coefficient $B_{12}$ $[m^{3}J^{-1}s^{-2}]$ where $12$ indicates the direction of process (i.e. $E_{1}$ to $E_{2}$) and $B$ indicates the process is stimulated (induced). After plugging the Einstein coefficient in the proportional equation, we get,

\begin{equation}
    (\frac{dn_{1}}{dt})_{st.a.} = -B_{12}n_{1}\rho(\nu) 
\end{equation}

To find the rate at which the atoms are getting excited to $E_{2}$ by stimulated absorption, we can substitute the above rate in the equation (4),

\begin{equation*}
    (\frac{dn_{2}}{dt})_{st.a.} = -(\frac{dn_{1}}{dt})_{st.a.} = B_{12}n_{1}\rho(\nu)
\end{equation*}

\subsubsection{Spontaneous Emission}

In this process, the atoms de-excite from $E_{2}$ to $E_{1}$ spontaneously (i.e. without any external aid). Like we did for stimulated absorption, we are going to consider $E_{1}$ (ground state) as our reference to find rate of this emission.
\begin{align*}
    \text{Rate of spontaneous} & = \text{Rate at which atoms are undergoing spontaneous} \\
    \text{emission} & \quad \text{emission from } E_{2} \text{ to } E_{1} \\ 
    & = (\frac{dn_{1}}{dt})_{sp.e.}
\end{align*}

And, from experiments, we know, 
\begin{align*}
    \textnormal{Rate of spontaneous emission} \propto n_{2} \\ 
    (\frac{dn_{1}}{dt})_{sp.e.} \propto n_{2}
\end{align*}

As atoms de-excite from $E_{2}$ to $E_{1}$, there will be an increase in $n_{1}$ (and decrease in $n_{2}$) which explains why it has no negative sign. The rate of spontaneous emission will be more if there are more excited atoms in $E_{2}$ (i.e. $n_{2}$) to get de-excited spontaneously, so the rate depends on $n_{2}$. As this process is independent of any external aid (i.e. no incident radiation like in stimulated absorption), the rate is independent of spectral energy density. \vspace{.2cm}

The proportional constant for the equation is the Einstein coefficient $A_{21}$ $[s^{-1}]$ where $21$ indicates the direction of process (i.e. $E_{2}$ to $E_{1}$) and $A$ indicates the process is spontaneous. After plugging the Einstein coefficient in the proportional equation, we get,
\begin{equation}
    (\frac{dn_{1}}{dt})_{sp.e.} = A_{21}n_{2}
\end{equation}

We can get the rate of this emission keeping $E_{2}$ as our reference by applying equation (4) like we did earlier in stimulated absorption.

\subsubsection{Stimulated Emission}

In this process, the atoms de-excite from $E_{2}$ to $E_{1}$ due to incident of radiation of frequency $\nu$. Taking $E_{1}$ (ground state) as our reference like we did earlier.
\begin{align*}
    \text{Rate of stimulated} & = \text{Rate at which atoms are undergoing stimulated} \\
    \text{emission} & \quad \text{emission from } E_{2} \text{ to } E_{1} \\ 
    & = (\frac{dn_{1}}{dt})_{st.e.}
\end{align*}

And, from experiments, we know, 
\begin{align*}
    \textnormal{Rate of stimulated emission} \propto n_{2}\rho(\nu) \\ 
    (\frac{dn_{1}}{dt})_{st.e.} \propto n_{2}\rho(\nu)
\end{align*}

Like in spontaneous emission, the atoms get de-excited from $E_{2}$ to $E_{1}$ making the rate dependent on the population density in excited state (i.e. $n_{2}$) and has a positive sign (due to the increase in $n_{1}$). Unlike in spontaneous emission, here the emission takes place because of the incident radiation, so the rate here depends on the spectral energy density of the incident radiation (like in stimulated absorption). \vspace{.2cm}

The proportional constant for the equation is the Einstein coefficient $B_{21}$ $[m^{3}J^{-1}s^{-2}]$ where $21$ indicates the direction of process (i.e. $E_{2}$ to $E_{1}$) and $B$ indicates the process is stimulated. After plugging the Einstein coefficient in the proportional equation, we get,
\begin{equation}
    (\frac{dn_{1}}{dt})_{sp.e.} = B_{21}n_{2}\rho(\nu)
\end{equation}

Again, we can get the rate of this emission keeping $E_{2}$ as our reference by applying equation (4) like we did earlier in stimulated absorption.

\subsection{Probability aspect of Einstein Coefficients}

\subsubsection{Stimulated absorption-only system}

Let's take a system where atoms only undergo stimulated absorption (no stimulated emission or spontaneous emission) from its $E_{1}$ state to $E_{2}$ state when an radiation of frequency $\nu$ is incident on it. \vspace{.2cm}

In that system, let $P(t)$ be the probability of an atom to undergo stimulated absorption from $E_{1}$ to $E_{2}$ at time $t$ and let $P^c(t)$ be the probability of an atom to remain in the same $E_{1}$ state at time $t$ (i.e. not undergoing any transition), \vspace{.2cm}
\begin{equation}
    P(t) + P^c(t) = 1
\end{equation}

As more and more atoms in $E_{1}$ undergo stimulated absorption to $E_{2}$, the population density $n_{1}$ decreases which means an atom in $E_{1}$ state has more chances (i.e. probability) to absorb the incident radiation and get excited to $E_{2}$. One can imagine it like this: if the supplies are fixed and the number of people in the crowd is decreasing, then it becomes more probable for each person in the crowd to obtain a supply. This explains why is the probability of an atom getting excited or not is a function of time $t$ (i.e. dependent on time $t$). \vspace{.2cm}

We can define the probability of an atom to remain in the same state (i.e. $P^c(t)$) as,
\begin{align}
    \text{Probability of an atom to} & = \text{Ratio of population density of atoms} \notag \\
    \text{remain in the state} & \quad \text{  in that state at t=t to the population} \notag \\
    & \quad \text{  population density at t=0} \notag \\
    \Rightarrow P^c(t) & = \frac{n_{1}(t)}{n_{1}(0)} 
\end{align}

Here, $n_{1}(t)$ is the population density at $E_{1}$ state at time $t$. \vspace{.2cm}

Since we assumed there is only stimulated absorption taking place in the system, the change in population density of atoms in $E_{1}$ is only due to stimulated absorption,

\begin{equation*}
    \frac{dn_{1}}{dt} = (\frac{dn_{1}}{dt})_{st.a.}
\end{equation*} \vspace{.1cm}

Now, we can plug equation (8) into it,
\begin{equation*}
    \frac{dn_{1}}{dt} = -B_{12}n_{1}\rho(\nu) 
\end{equation*}

After rearranging the terms, we can integrate the equation on the LHS from $n_{1}(0)$ to $n_{1}(t)$ and on the RHS from $0$ to $t$, where $n_{1}(t)$ is the population density $n_{1}$ at time $t$,
\begin{align}
    \Rightarrow \frac{dn_{1}}{n_{1}} & = -B_{12}\rho(\nu)dt \notag \\
    \Rightarrow \int_{n_{1}(0)}^{n_{1}(t)} \frac{dn_{1}}{n_{1}} & = \int_{0}^{t} -B_{12}\rho(\nu)dt \notag \\
    \Rightarrow ln(\frac{n_{1}(t)}{n_{0}(0)}) & = -B_{12}\rho(\nu)(t - 0) \notag \\
    \Rightarrow \frac{n_{1}(t)}{n_{1}(0)} & = e^{-B_{12}\rho(\nu)t}
\end{align}

Now, we can substitute the equation (12) into it,
\begin{equation*}
    P^c(t) = e^{-B_{12}\rho(\nu)t}
\end{equation*}

Using equation (11), we can find the value of $P(t)$,
\begin{align}
    P(t) & = 1 - P^c(t) \notag \\
    \Rightarrow P(t) & = 1 - e^{-B_{12}\rho(\nu)t}
\end{align}

Thus, we have arrived at the equation for the probability of an atom undergoing stimulated absorption . \textbf{Note} that this has been derived under the assumption that no other interactions (such as stimulated emission or spontaneous emission) are taking place. In reality (e.g., in the lasing process), all three interactions (i.e. stimulated absorption, stimulated emission, and spontaneous emission) take place, so in such cases, we cannot apply this equation. \vspace{.2cm}

Interactions like stimulated absorption usually occur almost instantly, so we can consider $t \rightarrow 0$. To simplify our equation further, we can use the limit $t \rightarrow 0$, for which we will divide our equation with $t$ first,

\begin{equation*}
    \Rightarrow \frac{P(t)}{t} = \frac{1 - e^{-B_{12}\rho(\nu)t}}{t}
\end{equation*} \vspace{.05cm}

Now, apply limits only on the RHS of the equation,
\begin{align}
    \Rightarrow \frac{P(t)}{t} &= \lim_{t \to 0} \frac{1 - e^{-B_{12}\rho(\nu)t}}{t} \notag \\
    \Rightarrow \frac{P(t)}{t} &= B_{12}\rho(\nu) \notag \\
    \Rightarrow P(t) &= B_{12}\rho(\nu)t
\end{align} 

Usually, when applying limits, we have to apply it on both sides not just on one side of the equation. However, in this case, we were trying to simplify by approximation. \textbf{Note} that if $t$ is significant enough, this equation fails because of the approximation, so in such cases, we have to use equation (14). \vspace{.2cm}

\begin{note}
    \textbf{Time Saving Result} \vspace{.2cm}

    Exponential functions of form like in equation (14) are very common in Physics. They can be simplified by approximation if $x \to 0$ by applying limits on the RHS.
    \begin{align*}
        y &= 1-e^{kx} \\
        \Rightarrow y & \approx -kx \quad \because x \to 0
    \end{align*}
\end{note}

By rearranging the terms in equation (15), we can arrive at,
\begin{equation*}
    B_{12} = \frac{P(t)}{\rho(\nu)t}
\end{equation*}

Thus, we can define $B_{12}$ as the probability per unit time per unit energy density of the radiation field per unit frequency that an atom in $E_{1}$ absorb the incident radiation (of suitable energy) and get excited to $E_{2}$ in a system where no other interactions (processes) are taking place in a small interval of time. \vspace{.2cm}

A lot of books or articles (like Wikipedia), refers Einstein coefficients as "probability" and by "probability", this is what they mean. Truly speaking, Einstein coefficients are more of rate constants (which we can see even in the differential equations) than probability. As the value of einstein coefficient increases, the rate increases and as the rate increases, more atoms excite from $E_{1}$ to $E_{2}$ so there are more chances for an atom in $E_{1}$ to undergo stimulated absorption thus affecting the value of probability. So, even in a system where all kind of interactions takes place, the einstein coefficient is related to probability but not linearly proportional as derived above under several assumptions.

\subsubsection{Stimulated emission-only system}

Let's take a system where only atoms undergo stimulated emission from $E_{2}$ to $E_{1}$ when a radiation of frequency $\nu$ is incident on it. \vspace{.2cm}

In this system, let $P(t)$ be the probability of an atom to undergo stimulated emission from $E_{2}$ to $E_{1}$ at time $t$ and let $P^c(t)$ be the probability of an atom to remain in the same $E_{2}$ state at time $t$ (i.e. not undergoing any transition). \vspace{.2cm}

Unlike in absorption, where atoms move from $E_{1}$ to $E_{2}$, in emission, atoms move from $E_{2}$ to $E_{1}$, so we will be using $dn_{2}/dt$ and $n_{2}$ to obtain probability equation,
\begin{equation*}
    P^c(t) = \frac{n_{2}(t)}{n_{2}(0)}
\end{equation*}

$P^c(t)$ represents the probability of atoms that did not undergo the process. In the absorption process, $P^c(t)$ refers to the atoms in the state $E_1$ that did not undergo absorption. Conversely, in the emission process, $P^c(t)$ refers to the atoms in the state $E_2$ that did not undergo emission. Therefore, we use $n_2$ instead of $n_1$ to determine $P^c(t)$ in the emission process. \vspace{.2cm}

Since we assumed there is only stimulated emission taking place in the system, the change in population density of atoms in $E_{2}$ is only due to stimulated emission,
\begin{equation*}
    \frac{dn_{2}}{dt} = (\frac{dn_{2}}{dt})_{st.e.}
\end{equation*} \vspace{.1cm}

Here, we are finding the value of $dn_{2}/dt$ instead of $dn_{1}/dt$ because we are interested in the value of the fraction $n_{2}(t)/n_{2}(0)$. \vspace{.2cm}

As we are familiar with the math and the procedure, we can skip steps,
\begin{align*}
    \frac{dn_{2}}{dt} &= -B_{21}n_{2}\rho(\nu) \\
    \Rightarrow \frac{n_2(t)}{n_2(0)} &= e^{-B_{21}\rho(\nu)t} \quad \because \text{Integrating both sides} \\
    \Rightarrow P^c(t) &= e^{-B_{21}\rho(\nu)t} \\
    \Rightarrow P(t) &= 1 - e^{-B_{21}\rho(\nu)t} \quad \because P(t) = 1 - P^c(t) \\
    \Rightarrow P(t) & \approx B_{21}\rho(\nu)t \quad \because t \to 0
\end{align*}

Now, by rearranging the terms, we arrive at,
\begin{equation*}
    B_{21} = \frac{P(t)}{\rho(\nu)t}
\end{equation*}

Thus, we can define $B_{21}$ as the probability per unit time per unit energy density of the radiation field per unit frequency that an atom in $E_{2}$ gets stimulated by the incident radiation (of suitable energy) and get de-excited to $E_{1}$ emitting a radiation of an equal energy in a system where no other interactions (processes) are taking place in a small interval of time.

\subsubsection{Spontaneous emission-only system}

Let's take a system where only atoms undergo spontaneous emission from $E_{2}$ to $E_{1}$. In this system, let $P(t)$ be the probability of an atom to undergo spontaneous emission from $E_{2}$ to $E_{1}$ at time $t$ and let $P^c(t)$ be the probability of an atom to remain in the same $E_{2}$ state at time $t$ (i.e. not undergoing any transition). \vspace{.2cm}

Since spontaneous emission is a type of emission process similar to stimulated emission, with the key difference being that spontaneous emission does not require incident radiation to be stimulated and is thus independent of the spectral energy density of the incident radiation, we can modify the equation for the probability of stimulated emission by removing $\rho(\nu)$ and changing the Einstein coefficient $B_{21}$ to $A_{21}$. This modification yields the probability of spontaneous emission.
\begin{align*}
    \frac{dn_{2}}{dt} &= -A_{21}n_{2}) \\
    \Rightarrow \frac{n_2(t)}{n_2(0)} &= e^{-A_{21}t} \quad \because \text{Integrating both sides} \\
    \Rightarrow P^c(t) &= e^{-A_{21}t} \\
    \Rightarrow P(t) &= 1 - e^{-A_{21}t} \quad \because P(t) = 1 - P^c(t) \\
    \Rightarrow P(t) & \approx A_{21}t \quad \because t \to 0
\end{align*}

Now, by rearranging the terms, we arrive at,
\begin{equation*}
    A_{21} = \frac{P(t)}{t}
\end{equation*}

Thus, we can define $A_{21}$ as the probability per unit time per unit energy density of the radiation field per unit frequency that an atom in $E_{2}$ gets de-excited to $E_{1}$ emitting a radiation (of a suitable energy) in a system where no other interactions (processes) are taking place in a small interval of time.

\subsection{Relation between the Einstein Coefficients}

Let's consider system having two energy levels (like we did earlier) i.e., $E_{1}$ (ground state) and $E_{2}$ (excited state). If radiation with energy equal to $\Delta E$ between the energy states, then, all the three kind of interactions between radiation and matter will happen. So, at any point of time $t$,
\begin{align}
    \text{Rate of change in} & = \text{Rate at which atoms} - \text{Rate at which atoms} \notag \\
    \text{no. of atoms in } E_{i} & \quad \text{ are excited from } E_{i} \quad \qquad \text{are de-excited to } E_{i} \notag \\
    \Rightarrow \frac{dn_{i}}{dt} & = (\frac{dn_{i}}{dt})_{st.a.} - ((\frac{dn_{i}}{dt})_{sp.e.} + (\frac{dn_{i}}{dt})_{st.e.}) \notag
\end{align}

As we have considered, $E_{1}$ (ground state) to be our reference for rate equations, we will take $i = 1$ and plug-in the equations (8), (9), (10) in the above one,
\begin{align}
    \Rightarrow \frac{dn_{1}}{dt} & = (\frac{dn_{1}}{dt})_{st.a.} - ((\frac{dn_{1}}{dt})_{sp.e.} + (\frac{dn_{1}}{dt})_{st.e.}) \notag \\
    \Rightarrow \frac{dn_{1}}{dt} & = B_{12}n_{1}\rho(\nu) - (A_{21}n_{2} + B_{21}n_{2}\rho(\nu))
\end{align}

Now, we can consider Thermodynamic Equilibrium which will make $dn_{1}/dt = 0$ by using the equation (5), 
\begin{align}
    \Rightarrow 0 & = B_{12}n_{1}\rho(\nu) - (A_{21}n_{2} + B_{21}n_{2}\rho(\nu)) \notag \\
    \Rightarrow B_{12}n_{1}\rho(\nu) & = A_{21}n_{2} + B_{21}n_{2}\rho(\nu)
\end{align}

Thus, we have arrived at an relationship between the Einstein's coefficients at Thermodynamic Equilibrium. Considering $dn_{1}/dt = 0$ under equilibrium conditions is called as \textbf{Detailed Balancing}. \vspace{.2cm}

Also, we can further manipulate the equation to get the spectral energy density of the radiation,
\begin{align*}
    \Rightarrow \rho(\nu) = \frac{A_{21}n_{2}}{B_{12}n_{1} - B_{21}n_{2}} \\
    \Rightarrow \rho(\nu) = \frac{A_{21}}{B_{21}} \frac{1}{\frac{B_{12}n_{1}}{B_{21}n_{2}} - 1}
\end{align*}

As we have already assumed Thermodynamic Equilibrium, we can substitute the Boltzmann factor of equation (6) here,
\begin{equation*}
    \Rightarrow \rho(\nu) = \frac{A_{21}}{B_{21}} \frac{1}{\frac{B_{12}}{B_{21}} e^{-\frac{E_{1} - E_{2}}{kT}} - 1}
\end{equation*} \vspace{.2cm}

We are more interested with the frequency of incident radiation involved in the interaction (also spectral energy density is a function of that frequency), so we can substitute equation (2) here,
\begin{equation}
    \Rightarrow \rho(\nu) = \frac{A_{21}}{B_{21}} \frac{1}{\frac{B_{12}}{B_{21}} e^{\frac{hf}{kT}} - 1}
\end{equation}

Now, compare this equation with Planck's law in equation (7), we get,
\begin{align}
    \frac{A_{21}}{B_{21}} & = \frac{8\pi h \nu^3}{c^3} \\
    B_{12} & = B_{21}
\end{align}

\textbf{Note} that these relations are valid only under Thermodynamic Equilibrium conditions. \vspace{.2cm}

Now, $B_{12}$ and $B_{21}$ are same (independent of directions), we can drop the subscript notation for the coefficients when discussing equilibrium conditions, i.e., $A$ is the Einstein coefficient for spontaneous process and $B$ is the Einstein coefficient for stimulated process. With this, we can make equation (13) look more simple,
\begin{equation*}
    \rho(\nu) = \frac{A}{B} \frac{1}{e^{\frac{hf}{kT}} - 1}
\end{equation*} 

From (15), we can deduce that the probability of an atom in $E_{1}$ undergoing stimulated absorption is same as the probability of an atom in $E_{2}$ undergoing stimulated emission in equilibrium conditions. To put it simply, there is an equal probability for an atom to undergo stimulated emission or absorption by the incident radiation irrespective of its energy state under equilibrium conditions. \vspace{.2cm} 

From (14), we can get this proportional equation,
\begin{equation*}
    \frac{A_{21}}{B_{21}} \propto \nu^{3}
\end{equation*}

Here, we can deduce that for atomic transitions involving radiations of lower frequencies, stimulated emissions are more probable $(A_{21} << B_{21})$ and for atomic transitions involving radiations of higher frequencies, spontaneous emissions are more probable $(A_{21} >> B_{21})$. Now, substituting (2) in that proportional equation gives us,
\begin{equation}
    \frac{A_{21}}{B_{21}} \propto (\Delta E)^{3} \quad \because \nu = \frac{\Delta E}{h}
\end{equation}

From this, we can deduce that (same logic as earlier) if the energy difference between the energy levels is too low, it is more probable for the atom to undergo stimulated emission $(A_{21} << B_{21})$ and if the energy difference is too high, it is more probable for the atom to undergo spontaneous emission $(A_{21} >> B_{21})$. \vspace{.2cm}

This makes sense because, the instability of an excited atom increases with energy (i.e. $\Delta E$) absorbed by it to get excited. The instability of an atom makes the atom to de-excite faster without any external aid to make itself stable which makes it probable for spontaneous emission to occur. \vspace{.2cm}

To understand how high is considered high and how low is considered low, we can modify Planck's law in equation (7) by substituting equation (14) into it,
\begin{align}
    \rho(\nu) & = \frac{A_{21}}{B_{21}} \frac{1}{e^\frac{h\nu}{kT} - 1} \notag \\
    \Rightarrow \rho(\nu) \frac{B_{21}}{A_{21}} & = \frac{1}{e^\frac{h\nu}{kT} - 1} \notag \\
    \Rightarrow \frac{1}{\rho(\nu)} \frac{A_{21}}{B_{21}} & = e^\frac{h\nu}{kT} - 1
\end{align} \vspace{.2cm}

From equation (17), we can get the idea of high and low,
\begin{quote}
    \textbf{1) If $\Delta E = h\nu >> kT$,}
    \begin{align*}
        e^{\frac{h\nu}{kT}} & >> 1 \\
        \Rightarrow e^{\frac{h\nu}{kT}} - 1 & >> 1 \\
        \Rightarrow \frac{1}{\rho(\nu)} \frac{A_{21}}{B_{21}} & >> 1 \\
        \Rightarrow A_{21} & >> B_{21}
    \end{align*}

    Spontaneous emission becomes more probable than stimulated emission. \vspace{.2cm}

    \textbf{2) If $\Delta E = h\nu << kT$,}
    \begin{align*}
        e^{\frac{h\nu}{kT}} & << 1 \\
        \Rightarrow e^{\frac{h\nu}{kT}} - 1 & << 1 \\
        \Rightarrow A_{21} & << B_{21}
    \end{align*}

    Stimulated emission becomes more probable than spontaneous emission.
\end{quote}

Note that if $A_{21}$ and $B_{21}$ becomes comparable, we cannot say which will be more probable as it will depend on the complete probability equation (and several factors) which we will be discussing after Quantum Physics.

\section{Production of Lasers}

\end{document}