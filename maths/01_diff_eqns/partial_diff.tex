\documentclass[12pt]{article}
\usepackage{graphicx}
\usepackage[colorlinks=true, pdfborder={0 0 0}, linkcolor=blue, urlcolor=blue, citecolor=blue]{hyperref}
\usepackage[a4paper, margin=1in]{geometry}
\usepackage{float}
\usepackage{amsmath}
\usepackage{amssymb}
\usepackage{anysize}
\usepackage{caption}
\usepackage{tcolorbox}
\usepackage{titlesec}
\usepackage[normalem]{ulem}
\usepackage{enumitem}

\titleformat{\section}{\normalfont\LARGE\bfseries}{\thesection}{1em}{}
\titleformat{\subsection}{\normalfont\Large\bfseries}{\thesubsection}{1em}{}
\titleformat{\subsubsection}{\normalfont\large\bfseries}{\thesubsubsection}{1em}{}

\titlespacing*{\subsubsection}{0pt}{.8em}{.3em}

\renewcommand{\ULthickness}{1pt}
\renewcommand{\ULdepth}{0.5ex}  

\begin{document}
  

\fontsize{14}{16}
\selectfont
\captionsetup[figure]{font=normal}
\newtcolorbox{note}[1][]{colback=white, colframe=black, sharp corners, boxrule=0.3mm, #1}
\newtcolorbox{block}[1][left=1cm, top=-.2cm, bottom=-.2cm]{colback=white, colframe=white, sharp corners, boxrule=0.3mm, #1}

\title{Partial Derivatives}
\author{Sudarsan D Naidu}
\maketitle

\tableofcontents

\section{Introduction}

\subsection{Partial Derivative}

Partial Derivative represents how a multivariate function changes with respect to one variable while keeping all other variables constant. It is represented by $\frac{\partial}{\partial x}$ \vspace{.2cm}

Now, let's consider a multivariate function $f(x_1, x_2, ..., x_n)$ which depends on $n$ independent variables. We can express the partial derivation of the function with respect to the variable $x_i$ as a limit, \vspace{.2cm}
\begin{align}
    & \hspace{-\leftmargin} \frac{\partial}{\partial x_i}f(x_1, x_2, ..., x_i, ..., x_n) \notag \\
    & = \lim_{h \to 0} \frac{f(x_1, x_2, ..., x_i + h, ..., x_n) - f(x_1, x_2, ..., x_i, ..., x_n)}{h}
\end{align} \vspace{.2cm}

This partial derivative can also be represented as $\frac{\partial f}{\partial x_i}$, $\partial_{x_i} f$ or $f'_{x_i}$.

\subsection{Total Derivative}

Total derivatives represents how a multivariate function changes with respect to one variable considering the changes in all variables that the function depends on. It is represented by $\frac{d}{dx}$ \vspace{.2cm}

We can express the total derivative of $f$ with respect to a parameter $x$ using the partial derivatives with respect to each of its variables,
\begin{align}
    \frac{d}{dx}f(x_1, x_2, ..., x_n) = \frac{\partial f}{\partial x_1} \frac{dx_1}{dx} + \frac{\partial f}{\partial x_2} \frac{dx_2}{dx} + ... + \frac{\partial f}{\partial x_n} \frac{dx_n}{dx}
\end{align}

One can prove this by simply reverse engineering the chain rule of derivatives. \vspace{.2cm}

\subsubsection{Total Derivative w.r.t an Independent variable}

If the $n$ variables of the function $f$ are independent of each other, then the total derivative of the function with respect to any of it's variable will be same as it's partial derivative.

\begin{note}
    \textbf{Proof:} \vspace{.2cm}

    Let's consider a function $f(x_1, x_2, ..., x_n)$ depending on $n$ independent variables,

    \begin{equation*}
        \frac{dx_i}{dx_j} = 0 \text{,  } \forall \text{ } 0 < i, j \leq n \text{, } i \neq j
    \end{equation*}

    Now applying the total derivative formula, to find the total derivative of $f$ with respect to $x_i \text{ } (0 < i \leq n)$,

    \begin{align*}
        \frac{df}{dx_i} &= \frac{\partial f}{\partial x_1} \frac{dx_1}{dx_i} + \frac{\partial f}{\partial x_2} \frac{dx_2}{dx_i} + ... + \frac{\partial f}{\partial x_i} \frac{dx_i}{dx_i} + ... + \frac{\partial f}{\partial x_n} \frac{dx_n}{dx_i} \\
        \Rightarrow \frac{df}{dx_i} &= \frac{\partial f}{\partial x_1} \cdot 0 + \frac{\partial f}{\partial x_2} \cdot 0 + ... + \frac{\partial f}{\partial x_i} \cdot 1 + ... + \frac{\partial f}{\partial x_n} \cdot 0 \\
        \Rightarrow \frac{df}{dx_i} &= \frac{\partial f}{\partial x_i}
    \end{align*}
\end{note}

\subsection{Derivative of Univariate Functions}

So far, we have only been dealing with univariate functions (functions which depend only on one variable) that depend on a single variable, so there was no need for 'partial derivatives' and 'total derivatives,' as both of them are the same and are simply referred to as 'derivatives'. \vspace{.2cm}

\begin{note}
    \textbf{Proof:} \vspace{.2cm}

    Applying the total derivative formula for an univariate function $p(x)$,
    \begin{align*}
        \frac{dp}{dx} & = \frac{\partial p}{\partial x} \frac{dx}{dx} \\
        \Rightarrow \frac{dp}{dx} & = \frac{\partial p}{\partial x}
    \end{align*}
\end{note} \vspace{.2cm}

We can think of the 'derivatives' of the univariate functions we've been dealing with so far as the partial derivatives with respect to their only variable.

\section{Equality of Mixed Partials}

Exchanging the order of partial derivatives of a multivariate function does not change the result if certain conditions are satisfied. See \href{https://en.wikipedia.org/wiki/Symmetry_of_second_derivatives}{Wikipedia}.

\begin{equation}
    \frac{\partial}{\partial x_i} (\frac{\partial f}{\partial x_j}) = \frac{\partial}{\partial x_j} (\frac{\partial f}{\partial x_i})
\end{equation} \vspace{.2cm}

The mixed partial $\frac{\partial}{\partial x_i} (\frac{\partial f}{\partial x_j})$ can also be represented as,

\begin{equation*}
    \frac{\partial^2 f}{\partial x_i \partial x_j} \text{, } \partial_{x_i}\partial{x_j} f \text{ or } f'_{x_{i}x_{j}}
\end{equation*}

\section{Integrating Partial Derivatives}

We have learnt that derivative and integral cancels out for a univariate function. Take $p(x)$,

\begin{equation*}
    \int \frac{dp}{dx} \text{ } dx = p(x) + c
\end{equation*} \vspace{-.2cm}

where, $c$ is some constant (i.e. independent of x). This is called as \textbf{First Fundamental Theorem of Calculus}. \vspace{2cm}

When taking about multivariate functions, it is better to say that partial derivative and integral cancels out. Take $f(x_1, ..., x_i, ..., x_n)$,
\begin{equation*}
    \int \frac{\partial}{\partial x_i} f(x_1, ..., x_n) \text{ } dx_i = f(x_1, ..., x_n) + C(x_1, ..., x_{i-1}, x_{i+1}, ..., x_n)
\end{equation*}

where $C(x_1, ..., x_{i-1}, x_{i+1}, ..., x_n)$ is a function which depends on all variables on $f$ except for $x_i$ (i.e. independent of $x_i$ or constant with respect to $x_i$). \vspace{.2cm}

The above equation can be taken as a generalised form of \textbf{First Fundamental Theorem of Calculus} for a multivariate function.

\end{document}